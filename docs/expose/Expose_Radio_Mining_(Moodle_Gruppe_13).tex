\documentclass[ngerman,a4paper,11pt,parskip=half]{scrartcl}
\usepackage[utf8]{inputenc}
\usepackage[T1]{fontenc}
\usepackage{babel}
\usepackage{hyperref}
\usepackage{csquotes}
\usepackage{enumitem}
\usepackage{amssymb}
\usepackage{amsmath}
\usepackage{array}
\usepackage{booktabs}
\usepackage{geometry}
\usepackage[protrusion=true, expansion=true]{microtype}
\geometry{left=25mm,right=25mm,top=30mm,bottom=30mm}

% -------------------------
% Makros
\newcommand{\datasource}[2]{\item \textbf{#1:} #2}
% -------------------------

\title{Expos\'e\\RadioMining: Eine Analyse von Radiosendern und den deutschen Charts.}
\author{Projektgruppe 13 (Moodle Gruppe 13)}
\date{15.~Juni~2025}

\begin{document}
\maketitle

\section*{Kurze Projektbeschreibung}
Das Projekt \textbf{\glqq RadioMining\grqq{}} untersucht einige Radiosender aus dem DACH-Raum, sowie deren
Zusammenhang mit den offiziellen Deutschen Charts. Im Detail werden die Wiedergabe\- bzw. Playlisten, 
sowie die Startseiten (engl. landing page) der Radiosender untersucht. Hierzu wurde ein GitHub-Projekt 
erstellt\footnote{\url{https://github.com/fanonwue/fhswf-radio-scraper}}. \newline
Zustätzlich werden auch gesprochene Inhalte aus Radiosendungen aufgezeichnet, transkripiert und anaylsiert.
Hierzu wurde das Tool \texttt{audio\_miner}\footnote{\url{https://github.com/smilchsack/audio_miner}} 
erstellt,  welches mittels \texttt{ffmpeg} die Livestreams der Sender lokal speichert. Im Anschluss werden diese mittels 
\texttt{PyTorch} und \texttt{pyannote}\footnote{\url{https://huggingface.co/pyannote}} 
einer simplen Sprechererkennung durchgeführt und dann der augefzeichnete Ton mithilfe von
\texttt{OpenAI~Whisper}\footnote{\url{https://github.com/openai/whisper}} transkribiert.\newline
Ein \texttt{Streamlit}-gestütztes Tagging-Tool 
erlaubt das komfortable Labeln jeder Transkriptzeile 
mit Kategorien wie \texttt{music}, \texttt{traffic} oder \texttt{advertisement}. 
Anschließend trainiert ein \texttt{spaCy}-Klassifikator die automatische Einordnung neuer Segmente.\newline
Zuletzt möchte das Projekt prüfen, inwieweit es möglich ist, die mittels Audio\-Mining erfassten Informationen
mit den per Webscraping erfassten Informationen zu kombinieren.
\newline

\section*{Projektziele und Scope}
\begin{itemize}[noitemsep,leftmargin=*]
  \item \textbf{Datenerhebung:} Kontinuierliches Scraping der Playlists und Landingpages, 
  sowie der Livestreams von einigen der Radiosender.
  \item \textbf{Transkription \& Tagging:} Batch-Transkription mit 
  \texttt{audio\_miner} und manuelle Annotation mittels des Tagging-Tools.
  \item \textbf{Modelltraining:} Training mittels eines \texttt{spaCy}-Textklassifikators zur automatischen 
  Segmentklassifikation.
  \item \textbf{Vergleich mit Charts:} Abgleich der ermittelten Musikrotationen mit den Top~100 
  der offiziellen Charts, zur Analyse der Sendernähe zu Mainstream-Trends.
    \item \textbf{Vergleich von Nachrichten Überschriften und Transkripten:} Untersuchung der 
    gesprochenen Inhalte
  der Radiosender im Vergleich zu den Nachrichtenüberschriften der Radiosender.
  \item \textbf{Visualisierung \& Bericht:} Auswertung der Ergebnisse, Visualisierung der Daten 
  und Erstellung eines Abschlussberichts.
\end{itemize}

\section*{Genutzte Datenquellen}
  Die nachfolgende Tabelle listet die genutzten Datenquellen auf, die für das Projekt verwendet werden:

\begin{table}[h]
\centering
\small
\caption{Erfasste Datenquellen}
\begin{tabular}{lccc}
\textbf{Sender} & \textbf{Audio (Livestream)} & \textbf{Playlists} & \textbf{Landingpage} \\\hline
\href{https://www.swr.de/swr1/rp/index.html}{SWR1~RLP} & \checkmark & \checkmark & \checkmark \\
\href{https://www.swr3.de/}{SWR3} & \checkmark & \checkmark & \checkmark \\
\href{https://www.srf.ch/radio-srf-3}{SRF3} & -- & \checkmark & \checkmark \\
\href{https://www1.wdr.de/radio/wdr2/musik/playlist/index.jsp}{WDR2} & \checkmark & \checkmark & -- \\
\href{https://www.deutschlandfunknova.de/playlist}{DLF~Nova} & -- & \checkmark & -- \\
\href{https://www.radiomk.de/musik/playlist.html}{Radio~MK} & -- & \checkmark & -- \\
\href{https://www.offiziellecharts.de/charts/single}{Offizielle Charts} & -- & \checkmark & -- \\\hline
\end{tabular}
\\[2mm]\emph{-- = keine Erfassung, \checkmark = Erfassung erfolgt.}
\end{table}

\section*{Vorgehensweise}
Nachfolgend wird die Vorgehensweise des Projekts beschrieben. Das Projekt ist in vier Phasen unterteilt, 
die jeweils unterschiedliche Aufgaben und Ziele umfassen: 1 – Entwicklung, 2 – Datenerhebung, 3 – Analyse 
und 4 – Auswertung.

\newcounter{task}
\newcommand{\nr}{\stepcounter{task}\arabic{task}}

\begin{table}[h]
  \small
  \centering
  \caption{Aufgabenübersicht für das Projekt \textit{RadioMining}}
  \begin{tabular}{@{}cp{6.8cm}p{3.7cm}c@{}}
    \toprule
    \textbf{Nr.} & \textbf{Aufgabe} & \textbf{Bearbeiter} & \textbf{Phase} \\ \midrule
    \nr  & Exposé erstellen                                               & Sebastian Milchsack                                     & 1 \\
    \nr  & Web-Crawler-Basis \& \texttt{audio\_miner} entwickeln          & Fabian Wünderich, Sebastian Milchsack                   & 1 \\
    \nr  & Web-Crawler erstellen                                          & Sebastian Dornack, Sebastian Milchsack, Nils Robinet    & 1 \\
    \nr  & Crawler deployen u. Daten sammeln                              & Sebastian Dornack                                       & 2 \\
    \nr  & Audiostreams deployen, Daten sammeln u. transkribieren         & Sebastian Milchsack                                     & 2 \\
    \nr  & Audiostreams für n-Tage taggen                                 & TBD                                                     & 3 \\
    \nr  & Datenvorbereitung (Landingpages, Playlists)                    & TBD                                                     & 3 \\
    \nr  & spaCy-Klassifikator trainieren                                 & TBD                                                     & 3 \\
    \nr  & Musikrotationen $\leftrightarrow$ Top 100 abgleichen           & TBD                                                     & 4 \\
    \nr  & News-Titel $\leftrightarrow$ Transkripte prüfen                & TBD                                                     & 4 \\
    \nr  & Zwischenpräsentation                                           & gesamtes Team                                           & 4 \\
    \nr  & Ergebnisse evaluieren                                          & TBD                                                     & 4 \\
    \nr  & Resultate visualisieren, Endpräsentation u. Bericht erstellen  & gesamtes Team                                           & 4 \\ 
    \bottomrule
  \end{tabular}
\end{table}

\end{document}